\section*{Abstract}
\addcontentsline{toc}{section}{Abstract}
Humanoid robotics is a fast-developing field encompassing a wide range of engineering disciplines to solve the challenges of robotic interaction in environments designed for humans. In order to encourage this development many competitions exist around very common but highly specific human tasks. These all aim to extend the field of humanoid robotics and general intelligence.
In order to be a humanoid robot, a key feature is the ability to walk, robots need some method of actuating joints in a controllable, fast and energy efficient way.\newline
This project is to design and construct a single leg, based on a custom developed humanoid robotic platform. Specifically focusing on the knee and ankle joints. A single degree of freedom test rig has been constructed to test the feasibility of both the pneumatic muscles and the control scheme.
A mathematical model for the actuator has been used to inform the control system structure and allow the estimation of the current system state. Pneumatic muscles are a highly nonlinear actuator, this means that standard linear controllers have limited success in controlling the actuator position. Control can be achieved with only a static force map leading to an accurate, high-performance control scheme, however, pneumatic muscles also have their own dynamics, like hysteresis, some thermodynamic effects. \newline
Therefore, a nonlinear MPC controller has been implemented and theoretically compared to two other standard controllers. \newline