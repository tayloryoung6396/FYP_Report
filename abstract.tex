\section*{Abstract}
\addcontentsline{toc}{section}{Abstract}
Humanoid robotics is a fast-developing field encompassing a wide range of engineering disciplines to solve the challenges of robotics interaction in an environment designed for humans. In order to encourage this development many competitions exist around very common but highly specific human tasks, all of which aim to extend the field of humanoid robotics and general intelligence. In order to walk, robots need some method of actuating joints in a controllable, fast and energy efficient way. This project is to design and construct a single leg based on a custom developed humanoid robotic platform, focusing on the knee and ankle joints specifically. A mathematical model for the actuator will be used to inform the control system structure and allow the estimation of the current system state. The performance criteria for the actuator will be compared against two currently used servos. Pneumatic muscles are a highly nonlinear actuator, this means that standard linear controllers have limited success in controlling the actuator position. Control can be achieved with only a static force map leading to an accurate, high-performance controller scheme, however, pneumatic muscles also have their own dynamics, like hysteresis, some thermodynamic effects. 
Whilst attempting to perform these tests the actuators designed an implemented showed a less than desirable lifetime for inflation/deflation cycles. Three actuators failed within 10 cycles of inflation. This is expected to be due to an incompatibility between the inner bladder and the braided mesh. Moving forward different bladders will be tested to determine an effective compatibility combination by varying the material, diameter and thickness of the selected bladder.